\documentclass[12pts,a4paper]{article}
%cmd + shift + P
\usepackage{fontspec}   %加這個就可以設定字體
\usepackage{csvsimple}
\usepackage{geometry}
 \geometry{
 a4paper,
 total={170mm,257mm},
 left=20mm,
 top=20mm,
 }
\usepackage{xeCJK}       %讓中英文字體分開設置
\usepackage[T1]{fontenc}
\usepackage{bigfoot} % to allow verbatim in footnote
\usepackage[numbered,framed]{matlab-prettifier}
\usepackage{graphicx}
\usepackage{filecontents}
\graphicspath{ {img_hw1/} }
\setCJKmainfont{標楷體} %設定中文為系統上的字型,而英文不去更動,使用原TeX字型
\XeTeXlinebreaklocale "zh"             %這兩行一定要加,中文才能自動換行
\XeTeXlinebreakskip = 0pt plus 1pt     %這兩行一定要加,中文才能自動換行
\title{\textbf{生醫工程實驗, Spring 2017}\\ Lab3 超音波醫學影像分析(Ultrasound)}
\author{第五組\ 蔡承佑\ 許博竣\ 許秉鈞}
\date{May 24, 2017} %日期
\let\ph\mlplaceholder % shorter macro
\lstMakeShortInline"

\lstset{
  style              = Matlab-editor,
  basicstyle         = \mlttfamily,
  escapechar         = ",
  mlshowsectionrules = true,
}

\begin{document}
\maketitle
\section{HW1-Point Spread Function(PSF)}
\subsection{實驗目的}
了解超音波醫學影像的原理,並學習操作儀器測量仿體,量測各種不同模式下所觀測到的影
\subsection{實驗原理}
 超音波係利用電壓訊號經由壓電材料轉換成超音波,再感測反彈的超音波轉換成電訊號,藉 由不同的組織會有不同的聲阻抗,導致不一樣的反射係數,來進行感測,利用接收反射波的 時間為距離的兩倍除以速度來推算物體的遠近。
超音波可分為數種模式:A mode、B mode、M mode、Color Doppler Mode。A mode是將 反射波強度以振幅顯示;B mode是將反射波強度以亮度表示;M mode則是可以記錄連續時 間下空間的變化;Color Doppler Mode是利用都卜勒效應來量測移動物體的移動速度,並利 用顏色的方式呈現。 而由於需要對於特定的區域達到良好的解析度,需要聚焦射出的超音波,對於陣列感測器而 言,可利用不同時間發射超音波的方式,藉由相互干涉達成聚焦亦或是轉向的目的,稱為 beamformation。而接收反射時,在訊號中加入delay及weighting再相加,利用調整兩參數達 成動態聚焦的效果,以加大高解析度的範圍。
\subsection{實驗步驟}
1.用B mode,比較在res與pen模式,不同B mode gain 之標的物大小。\\
2.用B mode,比較在res與pen模式,不同B mode gain 下的雜訊。\\
3.用B mode,比較在res與pen模式下的CNR。\\ 4.用都卜勒模式測量人體的大動脈,截取其跳動影像及速率影像。

\subsection{實驗數據}
\textbf{a) Estimate PSF size for In-focused and Out-focused targets}  \\
以下用 5.7cm, 7.1cm 深度、 pen/res 模式來測量\\
\\
\textbf{b) Repeat (a) under different B-mode gain} \\

\pagebreak
\begin{itemize}
\item{\textbf{5.7cm/high gain/pen}}
\begin{center}
\csvautotabular{img_hw1/57-high-pen.csv}
\end{center}
\begin{figure}[h]
    \centering
    \includegraphics[width=1.0\textwidth]{img_hw1/57-high-pen1.jpg}
    \caption{57-high-pen(In Focus)}
    \label{fig:mesh1}
\end{figure}
\pagebreak
\begin{figure}[h]
    \centering
    \includegraphics[width=1.0\textwidth]{img_hw1/57-high-pen2.jpg}
    \caption{57-high-pen(Out Focus)}
    \label{fig:mesh1}
\end{figure}
\pagebreak

\item{\textbf{5.7cm/high gain/res}}
\begin{center}
\csvautotabular{img_hw1/57-high-res.csv}
\end{center}
\begin{figure}[h]
    \centering
    \includegraphics[width=1.0\textwidth]{img_hw1/57-high-res1.jpg}
    \caption{57-high-res(In Focus)}
    \label{fig:mesh1}
\end{figure}
\pagebreak
\begin{figure}[h]
    \centering
    \includegraphics[width=1.0\textwidth]{img_hw1/57-high-res2.jpg}
    \caption{57-high-res(Out Focus)}
    \label{fig:mesh1}
\end{figure}
\pagebreak
\item{\textbf{5.7cm/low gain/pen}}
\begin{center}
\csvautotabular{img_hw1/57-low-pen.csv}
\end{center}
\begin{figure}[h]
    \centering
    \includegraphics[width=1.0\textwidth]{img_hw1/57-low-pen1.jpg}
    \caption{57-low-pen(In Focus)}
    \label{fig:mesh1}
\end{figure}
\pagebreak
\begin{figure}[h]
    \centering
    \includegraphics[width=1.0\textwidth]{img_hw1/57-low-pen2.jpg}
    \caption{57-low-pen(Out Focus)}
    \label{fig:mesh1}
\end{figure}
\pagebreak
\item{\textbf{5.7cm/low gain/res}}
\begin{center}
\csvautotabular{img_hw1/57-low-res.csv}
\end{center}
\begin{figure}[h]
    \centering
    \includegraphics[width=1.0\textwidth]{img_hw1/57-low-res1.jpg}
    \caption{57-low-res(In Focus)}
    \label{fig:mesh1}
\end{figure}
\pagebreak
\begin{figure}[h]
    \centering
    \includegraphics[width=1.0\textwidth]{img_hw1/57-low-res2.jpg}
    \caption{57-low-res(Out Focus)}
    \label{fig:mesh1}
\end{figure}
\pagebreak
\item{\textbf{7.1cm/high gain/pen}}
\begin{center}
\csvautotabular{img_hw1/71-high-pen.csv}
\end{center}
\begin{figure}[h]
    \centering
    \includegraphics[width=1.0\textwidth]{img_hw1/71-high-pen1.jpg}
    \caption{71-high-pen(In Focus)}
    \label{fig:mesh1}
\end{figure}
\pagebreak
\begin{figure}[h]
    \centering
    \includegraphics[width=1.0\textwidth]{img_hw1/71-high-pen2.jpg}
    \caption{71-high-pen(Out Focus)}
    \label{fig:mesh1}
\end{figure}
\pagebreak
\item{\textbf{7.1cm/high gain/res}}
\begin{center}
\csvautotabular{img_hw1/71-high-res.csv}
\end{center}
\begin{figure}[h]
    \centering
    \includegraphics[width=1.0\textwidth]{img_hw1/71-high-res1.jpg}
    \caption{71-high-res(In Focus)}
    \label{fig:mesh1}
\end{figure}
\pagebreak
\begin{figure}[h]
    \centering
    \includegraphics[width=1.0\textwidth]{img_hw1/71-high-res2.jpg}
    \caption{71-high-res(Out Focus)}
    \label{fig:mesh1}
\end{figure}
\pagebreak
\item{\textbf{7.1cm/low gain/pen}}
\begin{center}
\csvautotabular{img_hw1/71-low-pen.csv}
\end{center}
\begin{figure}[h]
    \centering
    \includegraphics[width=1.0\textwidth]{img_hw1/71-low-pen1.jpg}
    \caption{71-low-pen(In Focus)}
    \label{fig:mesh1}
\end{figure}
\pagebreak
\begin{figure}[h]
    \centering
    \includegraphics[width=1.0\textwidth]{img_hw1/71-low-pen2.jpg}
    \caption{71-low-pen(Out Focus)}
    \label{fig:mesh1}
\end{figure}
\pagebreak
\item{\textbf{7.1cm/low gain/res}}
\begin{center}
\csvautotabular{img_hw1/71-low-res.csv}
\end{center}
\begin{figure}[h]
    \centering
    \includegraphics[width=1.0\textwidth]{img_hw1/71-low-res1.jpg}
    \caption{71-low-res(In Focus)}
    \label{fig:mesh1}
\end{figure}
\pagebreak
\begin{figure}[h]
    \centering
    \includegraphics[width=1.0\textwidth]{img_hw1/71-low-res2.jpg}
    \caption{71-low-res(Out Focus)}
    \label{fig:mesh1}
\end{figure}
\pagebreak
\end{itemize}

\textbf{HW1 CONCLUSION} \\
\\
大部分情況,In-focused 的大小都比 Out-focused 來的小,符合我們的預期, 除了少數情況之外,因為我們有些圖其實量的有些不清楚,可能造成在找 6dB 點的時,其真正範圍與觀測範圍不太一樣。
\section{HW2-Speckle Statistics}
在 B mode 下,分別以探頭可測最大深度 5.7cm 及 7.1 cm,量測在 high gain 及 low gain 下,比較 pen 與 res 時,in-focused 及 out-focused 的 standard variation,並作出 speckle intensity distribution 及 speckle amplitude distribution。

\begin{itemize}
\item{\textbf{5.7cm/high gain/pen}}
\begin{center}
\csvautotabular{img_hw2/57-high-pen.csv}
\end{center}
\begin{figure}[h]
    \centering
    \includegraphics[width=1.0\textwidth]{img_hw2/57-high-pen1.jpg}
    \caption{57-high-pen(In Focus)}
    \label{fig:mesh1}
\end{figure}
\pagebreak
\begin{figure}[h]
    \centering
    \includegraphics[width=1.0\textwidth]{img_hw2/57-high-pen2.jpg}
    \caption{57-high-pen(Out Focus)}
    \label{fig:mesh1}
\end{figure}
\pagebreak
\item{\textbf{5.7cm/high gain/res}}
\begin{center}
\csvautotabular{img_hw2/57-high-res.csv}
\end{center}
\begin{figure}[h]
    \centering
    \includegraphics[width=1.0\textwidth]{img_hw2/57-high-res1.jpg}
    \caption{57-high-res(In Focus)}
    \label{fig:mesh1}
\end{figure}
\pagebreak
\begin{figure}[h]
    \centering
    \includegraphics[width=1.0\textwidth]{img_hw2/57-high-res2.jpg}
    \caption{57-high-res(Out Focus)}
    \label{fig:mesh1}
\end{figure}
\pagebreak
\item{\textbf{5.7cm/low gain/pen}}
\begin{center}
\csvautotabular{img_hw2/57-low-pen.csv}
\end{center}
\begin{figure}[h]
    \centering
    \includegraphics[width=1.0\textwidth]{img_hw2/57-low-pen1.jpg}
    \caption{57-low-pen(In Focus)}
    \label{fig:mesh1}
\end{figure}
\pagebreak
\begin{figure}[h]
    \centering
    \includegraphics[width=1.0\textwidth]{img_hw2/57-low-pen2.jpg}
    \caption{57-low-pen(Out Focus)}
    \label{fig:mesh1}
\end{figure}
\pagebreak
\item{\textbf{5.7cm/low gain/res}}
\begin{center}
\csvautotabular{img_hw2/57-low-res.csv}
\end{center}
\begin{figure}[h]
    \centering
    \includegraphics[width=1.0\textwidth]{img_hw2/57-low-res1.jpg}
    \caption{57-low-res(In Focus)}
    \label{fig:mesh1}
\end{figure}
\pagebreak
\begin{figure}[h]
    \centering
    \includegraphics[width=1.0\textwidth]{img_hw2/57-low-res2.jpg}
    \caption{57-low-res(Out Focus)}
    \label{fig:mesh1}
\end{figure}
\pagebreak
\item{\textbf{7.1cm/high gain/pen}}
\begin{center}
\csvautotabular{img_hw2/71-high-pen.csv}
\end{center}
\begin{figure}[h]
    \centering
    \includegraphics[width=1.0\textwidth]{img_hw2/71-high-pen1.jpg}
    \caption{71-high-pen(In Focus)}
    \label{fig:mesh1}
\end{figure}
\pagebreak
\begin{figure}[h]
    \centering
    \includegraphics[width=1.0\textwidth]{img_hw2/71-high-pen2.jpg}
    \caption{71-high-pen(Out Focus)}
    \label{fig:mesh1}
\end{figure}
\pagebreak
\item{\textbf{7.1cm/high gain/res}}
\begin{center}
\csvautotabular{img_hw2/71-high-res.csv}
\end{center}
\begin{figure}[h]
    \centering
    \includegraphics[width=1.0\textwidth]{img_hw2/71-high-res1.jpg}
    \caption{71-high-res(In Focus)}
    \label{fig:mesh1}
\end{figure}
\pagebreak
\begin{figure}[h]
    \centering
    \includegraphics[width=1.0\textwidth]{img_hw2/71-high-res2.jpg}
    \caption{71-high-res(Out Focus)}
    \label{fig:mesh1}
\end{figure}
\pagebreak
\item{\textbf{7.1cm/low gain/pen}}
\begin{center}
\csvautotabular{img_hw2/71-low-pen.csv}
\end{center}
\begin{figure}[h]
    \centering
    \includegraphics[width=1.0\textwidth]{img_hw2/71-low-pen1.jpg}
    \caption{71-low-pen(In Focus)}
    \label{fig:mesh1}
\end{figure}
\pagebreak
\begin{figure}[h]
    \centering
    \includegraphics[width=1.0\textwidth]{img_hw2/71-low-pen2.jpg}
    \caption{71-low-pen(Out Focus)}
    \label{fig:mesh1}
\end{figure}
\pagebreak
\item{\textbf{7.1cm/low gain/res}}
\begin{center}
\csvautotabular{img_hw2/71-low-res.csv}
\end{center}
\begin{figure}[h]
    \centering
    \includegraphics[width=1.0\textwidth]{img_hw2/71-low-res1.jpg}
    \caption{71-low-res(In Focus)}
    \label{fig:mesh1}
\end{figure}
\pagebreak
\begin{figure}[h]
    \centering
    \includegraphics[width=1.0\textwidth]{img_hw2/71-low-res2.jpg}
    \caption{71-low-res(Out Focus)}
    \label{fig:mesh1}
\end{figure}
\end{itemize}

\textbf{HW2 CONCLUSION} \\
\\
由圖可以發現,intensity 跟 amplitude 大都符合理論的分布,standard deviation 也在理論範圍附近!
\pagebreak
\section{HW3-Contrast To Noise Ration(CNR)}
\subsection{實驗原理}
用於決定影像解析度高低的基準。強度可被視為訊號而標準差則為噪音。 CNR 的定義為
\begin{center}
$CNR=|\frac{I_{in}-I_{out}}{std_{in}-std_{out}}|$
\end{center}
\subsection{實驗數據}
\textbf{a) Estimate CNR at in-focused region and out-focused region} \\
在 B mode 下,分別以 high gain 及 low gain,量測 pen (in-focused) 及 res (out-focused) 的個點,並計算出其 CNR。

\begin{itemize}
\item{\textbf{high gain/pen}}
\begin{center}
\csvautotabular{img_hw3/high_pen.csv}
\end{center}
\begin{figure}[h]
    \centering
    \includegraphics[width=1.0\textwidth]{img_hw3/high_pen.jpg}
    \caption{high-pen}
    \label{fig:mesh1}
\end{figure}
\pagebreak
\item{\textbf{high gain/res}}
\begin{center}
\csvautotabular{img_hw3/high_res.csv}
\end{center}
\begin{figure}[h]
    \centering
    \includegraphics[width=1.0\textwidth]{img_hw3/high_res.jpg}
    \caption{high-res}
    \label{fig:mesh1}
\end{figure}
\pagebreak
\item{\textbf{low gain/pen}}
\begin{center}
\csvautotabular{img_hw3/low_pen.csv}
\end{center}
\begin{figure}[h]
    \centering
    \includegraphics[width=1.0\textwidth]{img_hw3/low_pen.jpg}
    \caption{low-pen}
    \label{fig:mesh1}
\end{figure}
\pagebreak
\item{\textbf{low gain/res}}
\begin{center}
\csvautotabular{img_hw3/low_res.csv}
\end{center}
\begin{figure}[h]
    \centering
    \includegraphics[width=1.0\textwidth]{img_hw3/low_res.jpg}
    \caption{low-res}
    \label{fig:mesh1}
\end{figure}
\pagebreak
\end{itemize}

\pagebreak
\section{HW4-Doppler}
\subsection{實驗原理}
在都卜勒模式下,以探頭測量頸動脈的血流,並擷取其 Color Doppler 及 PW 影像。
\subsection{實驗步驟}
\textbf{a) Carotid color Doppler and PW images} \\
\textbf{受測者:蔡承佑} \\
\begin{figure}[h]
    \centering
    \includegraphics[width=0.5\textwidth]{img_hw4/CPD.jpg}
    \caption{CPD}
    \label{fig:mesh1}
\end{figure}
\begin{figure}[h]
    \centering
    \includegraphics[width=0.5\textwidth]{img_hw4/CPD-2.jpg}
    \caption{CPD2}
    \label{fig:mesh1}
\end{figure}

\end{document}
